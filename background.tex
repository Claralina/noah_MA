\chapter{Background}\label{Background}
This chapter aims to provide a comprehensive overview of the key concepts, challenges, and advancements in privacy-preserving computation, specifically \ac{SMPC}, and record linkage.
\section{Secure Multi-Party Computation}

\subsection{Threat Models}
\subsection{Yao's Garbled Circuits}
\subsection{Arithmetic Sharing}
\subsection{Boolean Sharing}
\section{Record Linkage}
\subsection{Concept and Application}
Record linkage is the process of matching data points (records) related to the same entity, i.e. human, but originating from different datasets.
If a commonly shared unique identifier exists, this process is trivial. % TODO find example of this
For instance, the \ac{SSN} in the United States of America, in theory, serves this purpose.
In practice, even this is not perfect due to erroneous data.
More importantly, there are many instances where such a unique identifier is not readily available.
In these instances, the matching must be done on features of the entity that are found in both datasets. % TODO find multiple areas where record linkage is employed
The term record linkage in its broadest definition may be used to refer to any such process of linking datasets, such as %TODO find example (Arjun mentioned something?)
In this work, the term record linkage pertains to the linking of datasets using \ac{IDAT}.
These include information such as name, date of birth and place of residence of a patient.
In medical research, record linkage is commonly employed to match patient data from various healthcare institutions, creating a dataset that combines information from different sources.
Individually, each \ac{IDAT} point is insufficient to provide an unambiguous match, which is why they are often referred to as quasi-identifiers. % TODO where?
However, when combining enough quasi-identifiers, such a match can be achieved.
Issues in record linkage based on \ac{IDAT} arise when the data is either non-unique or inaccurate, which leads us to error types and their sources.

\Ac{IDAT} inaccuracies arise through various means, primarily during manual data entry.
Human input, typically via keyboard, introduces errors stemming from multiple sources.
Spelling mistakes may result from transcribing orally received information or copying text-based information.
Oral communication may introduce discrepancies due to similar-sounding words or varied spellings of the same name.
Copying text can lead to confusion between visually similar letters (e.g., I and l).
Typographical errors, induced by slips during typing, contribute to insertions, substitutions, transpositions, or deletions of letters.
Lastly, misplacement of individual attributes, such as swapping a patient's first name and surname, represents another source of error.

% TODO add section discussing research of spelling mistakes etc

\subsection{Classic Algorithms}
To tackle the complexities of
\subsection{Privacy-Preserving Algorithms}
